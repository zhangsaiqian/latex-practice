%% start of file `template-zh.tex'.
%% Copyright 2006-2012 Xavier Danaux (xdanaux@gmail.com).
%
% This work may be distributed and/or modified under the
% conditions of the LaTeX Project Public License version 1.3c,
% available at http://www.latex-project.org/lppl/.


\documentclass[11pt,a4paper,sans]{moderncv}   % possible options include font size ('10pt', '11pt' and '12pt'), paper size ('a4paper', 'letterpaper', 'a5paper', 'legalpaper', 'executivepaper' and 'landscape') and font family ('sans' and 'roman')

% moderncv 主题
\moderncvstyle{classic}                        % 选项参数是 ‘casual’, ‘classic’, ‘oldstyle’ 和 ’banking’
\moderncvcolor{blue}                          % 选项参数是 ‘blue’ (默认)、‘orange’、‘green’、‘red’、‘purple’ 和 ‘grey’
%\nopagenumbers{}                             % 消除注释以取消自动页码生成功能

% 字符编码
\usepackage[utf8]{inputenc}                   % 替换你正在使用的编码
\usepackage{CJKutf8}
%\usepackage{CJK}
% 调整页面出血
\usepackage[scale=0.75]{geometry}
%\setlength{\hintscolumnwidth}{3cm}           % 如果你希望改变日期栏的宽度

% 个人信息
\familyname{张赛谦}
\firstname{}

%\title{简历题目 (可选项)}                      % 可选项、如不需要可删除本行
%\address{大连理工大学}{}             % 可选项、如不需要可删除本行
\mobile{13591311934}                         % 可选项、如不需要可删除本行
%\phone{}                          % 可选项、如不需要可删除本行
%\fax{}                            % 可选项、如不需要可删除本行
\email{zhangsaiqian@gmail.com}                    % 可选项、如不需要可删除本行
%\homepage{weibo.com/u/1298310784}                  % 可选项、如不需要可删除本行
%\extrainfo{附加信息 (可选项)}                  % 可选项、如不需要可删除本行
%\photo[64pt][0.4pt]{picture}                  % ‘64pt’是图片必须压缩至的高度、‘0.4pt‘是图片边框的宽度 (如不需要可调节至0pt)、’picture‘ 是图片文件的名字;可选项、如不需要可删除本行
%\quote{引言(可选项)}                           % 可选项、如不需要可删除本行

% 显示索引号;仅用于在简历中使用了引言
%\makeatletter
%\renewcommand*{\bibliographyitemlabel}{\@biblabel{\arabic{enumiv}}}
%\makeatother

% 分类索引
%\usepackage{multibib}
%\newcites{book,misc}{{Books},{Others}}
%----------------------------------------------------------------------------------
%            内容
%----------------------------------------------------------------------------------
\begin{document}
\begin{CJK}{UTF8}{gbsn}                       % 详情参阅CJK文件包
\maketitle

\section{个人简介}
\cvitem{}{男,1987年12月生,河南省南阳市人,大连理工大学物理与光电工程学院博士研究生}

\section{教育背景}
\cventry{2010 -- 2015}{硕博连读}{}{大连理工大学}{等离子体物理}{}  % 第3到第6编码可留白
\cventry{2006 -- 2010}{理学学士}{}{大连理工大学}{应用物理}{}  % 第3到第6编码可留白


\section{研究方向}
\cvitem{}{等离子刻蚀剖面演化的模拟}
\cvitem{}{导师 戴忠玲 教授}
%\cvitem{说明}{\small 等离子体刻蚀}

\section{专业知识}
\subsection{应用物理(本科)}
\cvitem{专业课程}{数学、物理类,汇编语言,计算机硬件基础,计算物理,半导体材料物理,工程训练等}
%\cventry{}{}{}{}{}{a}
%\cventry{年 -- 年}{职位}{公司}{城市}{}{说明行1\newline{}说明行2}
\subsection{等离子体物理(研究生)}
\cvitem{专业课程}{等离子体物理前沿,等离子体技术与应用,论文写作与学术规范等}
\cvitem{专业技能}{在自己专业的学习中,了解研究背景,熟悉和掌握对物理问题建立数学模型,编程进行数值计算求解(流体方程、蒙特卡洛法),对结果进行数据分析和展示,并参与学术交流,包括国际会议上作英文口头报告}
%\cventry{年 -- 年}{职位}{公司}{城市}{}{说明}

\section{语言技能}

\cvitem{英语}{日常、学术交流口语,阅读文献,论文书写,大二527分过六级}{}
%\cvitemwithcomment{吴语}{江南长大}{能听懂}
%\cvitem{}{参与英语配音比赛等活动}{}

\section{计算机技能}
\cvdoubleitem{编程:}{C, FORTRAN, Python (matplotlib)}{文档排版:}{MS Office, \LaTeX, Vi}
\cvdoubleitem{网络:}{HTML基础, goagent, vpn}{计算机系统:}{GNU/Linux, OS X, Shell脚本}
\cvitem{软件:}{Visual Studio, COMSOL(多物理场仿真), Matlab, Origin, GCC, Gnuplot等}{}{}

\section{个人兴趣}
\cvitem{互联网}{个人博客,开源社区,微信微博知乎等}
\cvitem{电子数码}{手机、电脑、音乐器材、微单等}
%\cvitem{休闲娱乐}{\small 台球、羽毛球}

\section{所获奖项}
\cvlistitem{2009 学习优秀二等奖学金}
\cvlistitem{2010 研究生二等奖学金}
\cvlistitem{2013 国家奖学金}

\section{发表文章}
\cvlistitem{{\bfseries Zhang Saiqian}, Dai Zhongling and Wang Younian, \emph{Ion transport to a photoresist trench in a radio frequency sheath, 2012, Plasma Sci. Technol. 14 958}}
\cvlistitem{Zhong-Ling Dai, {\bfseries Sai-Qian Zhang}, You-Nian Wang, \emph{Study on Feature Profile Evolution for Chlorine Etching of Silicon in a RF Biased Sheath, Vacuum, Volume 89, March 2013, Pages 197–202}}
\cvlistitem{{\bfseries Sai-Qian Zhang}, Zhong-Ling Dai, Yuan-Hong Song, You-Nian Wang, \emph{Effect of reactant transport on the trench profile evolution for silicon etching in chlorine plasmas, Vacuum, Volume 99, January 2014, Pages 180–188}}

\renewcommand{\listitemsymbol}{-}             % 改变列表符号
%\section{联系方式}
%\cvitem{手机}{13591311934}
%\cvitem{邮箱}{zhangsaiqian@gmail.com}
%\section{其他 2}
%\cvlistdoubleitem{项目 1}{项目 4}
%\cvlistdoubleitem{项目 2}{项目 5\cite{book1}}
%\cvlistdoubleitem{项目 3}{}

% 来自BibTeX文件但不使用multibib包的出版物
%\renewcommand*{\bibliographyitemlabel}{\@biblabel{\arabic{enumiv}}}% BibTeX的数字标签
%\nocite{*}
%\bibliographystyle{plain}
%\bibliography{publications}                    % 'publications' 是BibTeX文件的文件名

% 来自BibTeX文件并使用multibib包的出版物
%\section{出版物}
%\nocitebook{book1,book2}
%\bibliographystylebook{plain}
%\bibliographybook{publications}               % 'publications' 是BibTeX文件的文件名
%\nocitemisc{misc1,misc2,misc3}
%\bibliographystylemisc{plain}
%\bibliographymisc{publications}               % 'publications' 是BibTeX文件的文件名

\clearpage\end{CJK}
\end{document}


%% 文件结尾 `template-zh.tex'.
